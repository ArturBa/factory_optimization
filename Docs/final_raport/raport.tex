\documentclass[a4paper]{article}
\usepackage[utf8]{inputenc}
\usepackage{amsmath}
\usepackage{polski}
\usepackage[polish]{babel}
\usepackage[T1]{fontenc}
\usepackage[a4paper,top=3cm,bottom=2cm,left=3cm,right=3cm,marginparwidth=1.75cm]{geometry}
\usepackage{graphicx}
\usepackage{float}
\usepackage{longtable}
\usepackage{pdflscape}
\usepackage[backend=bibtex]{biblatex}
\graphicspath{{../img/}}

\title{Optymalizacja fabryki z wykorzystaniem algorytmu immunologicznego (selekcji klonalnej)}
\author{Artur Bauer \and Kamil Szostek \and Sławomir Goździewski \and Wiktor Filipiak}
\date{\today}

\usepackage[pdftex,
            pdfauthor={Artur Bauer \& Kamil Szostek \& Sławomir Goździewski \& Wiktor Filipiak},
            pdftitle={\@title},
            pdfsubject={Glebokie uczenie i inteligencja obliczeniowa},
            pdfkeywords={Automatyka i Robotyka},
            pdfproducer={Latex},
            pdfcreator={pdflatex}]{hyperref}

\bibliography{citations.bib}

\begin{document}
%--------------------------------------------------------%
%	COVER PAGE
%--------------------------------------------------------%

\begin{titlepage}
\makeatletter

  \newcommand{\HRule}{\rule{\linewidth}{0.5mm}} % Defines a new command for the horizontal lines, change thickness here

  \center % Center everything on the page


%	HEADING SECTION

  \textsc{\LARGE Akademia Górniczo-Hutnicza}\\[1.5cm] % Name of your university/college
  \textsc{\Large  Głębokie uczenie i inteligencja obliczeniowa }\\[0.5cm] % Major heading such as course name
  \textsc{\large Automatyka i Robotyka II Stopień}\\[0.5cm] % Minor heading such as course title
  \textsc{2019/2020}\\[0.5cm] % Minor heading such as course title

%	TITLE SECTION

  \vspace{1.5 cm}
  \HRule \\[0.4cm]
  { \huge \bfseries \@title} \\[0.4cm] % Title of your document
  \HRule \\[1.5cm]
 
%	AUTHOR SECTION

  {\em\Large\textbf Skład zespołu:}\\
  \vspace{.5 cm}
    Artur Bauer\\
    Kamil Szostek\\
    Sławomir Goździewski\\
    Wiktor Filipiak
  \vspace{1.5 cm}
  
  
  {\em\Large\textbf Opiekun:}\\
  \vspace{.5 cm}
  dr hab. inż. Joanna Kwiecień
  
%	DATE SECTION

  \vspace{1.5 cm}
  {\large Złożono: \@date}\\[3cm] % Date, change the \today to a set date if you want to be precise


\vfill % Fill the rest of the page with whitespace

\end{titlepage}

%-----------------------------------------------------------------

\newpage

\tableofcontents

\newpage
\section{Wstęp}
\subsection{Model fabryki}\label{factory}

\begin{figure}[ht]
\centering
\includegraphics[width=.7\textwidth]{Factory_scheme.png}
\caption{Schemat fabryki}
\end{figure}

\subsubsection{Funkcja celu fabryki:}\label{factory-main-goal-function}

$$Income = \sum^{n_p}_{i=1}(p_i*(v_i-m_i*m_p)) - (1+b_i)*\sum^{n_w}_{i = 1}(w_i*s_i *t_{wi}) - m_r*m_p - punish$$

Gdzie:
\begin{itemize}
    \item $n_p$ -- ilość rodzajów części
    \item $p_i (n_m)$ -- ilość wyprodukowanych części i-tego typu
    \item $v_i(v_{bi}, t_{wi},t_{bi},w_q)$ -- wartość części i-tego typu
    \item $m_i$ -- liczba surowca potrzebna do wytworzenia elementu i-tego typu
    \item $m_p$ -- cena surowca
    \item $n_w$ -- liczba rodzajów pracowników
    \item $w_i$ -- liczba pracowników i-tego rodzaju
    \item $s_i$ -- wypłata pracownika i-tego rodzaju
    \item $b$ -- premia pracownicza
    \item $m_r(p_i,n_m)$ -- pozostały materiał
    \item $p_{i_{min}}$ -- minimalna ilość elementów do wytworzenia i uniknięcia kary
    \item $p_{i_{max}}$ -- maksymalna ilość wytworzonych elementów
    \item $n_m$ -- liczba surowca na początek dnia
\end{itemize}
\subsubsection{Kara}
$punish = p_{un}*\sum^{n_p}_{i=1}(p_{num_i})*v_i$

$p_{num_i}= \left\{\begin{matrix} 0  \;\quad\quad\quad\quad  \textrm{if} \quad  p_{i_{min}}-p_i \leq  0    \\ p_{i_{min}}-p_i  \quad  \textrm{if} \quad  p_{i_{min}}-p_i >  0  \end{matrix}\right.$

Gdzie: 
\begin{itemize}
    \item $p_{un}$ -- współczynnik kary
    \item $p_{num_i}(p_{i_{min}}, p_i)$ -- liczba elementów i-tego typu dla których naliczana jest kara
\end{itemize}
\subsubsection{Liczba pracowników}

Liczba pracowników i-tego typu jest równa ilości maszyn i-tego typu:

$n_p = n_w$

\subsubsection{Maksymalna ilość elementów}\label{max-number-of-items}

Niezbędna ilość elementów i-tego typu:

$\sum^{n_p}_{i=1} p_{i_{max}} * m_i< n_m$

\subsubsection{Rzeczywisty czas pracy maszyny na 1 produkt}\label{real-working-time-of-the-i-type-machineemployee-for-one-product} 
$t_{wi} = t_{pi} + p_i * t_{bi}$

\subsection{Parametry modelu}\label{model-assumption}

\begin{longtable}[c]{lll}
Parametr & oznaczenie & wartość\\ \hline
Ilość surowców & $n_m$ & [$x$ - 100]\\
Koszt surowca & $m_p$ & 15\\
Czas pracy & $t_f$ & [1 - 16]\\
Minimalna ilość dużych części & $p_{0_{min}}$ & [0 - 10]\\
Minimalna ilość małych części & $p_{1_{min}}$ & [0 -10]\\
Wypłata operatora dużej maszyny & $s_0$ & 70\\
Wymagana ilość materiału na duży element & $m_0$ & 4\\
Czas przygotowania dużej maszyny & $t_{p0}$ & 30 min\\
Wartość dużego elementu & $v_{b0}$ & 50\\
Podstawowy czas pracy na duży element & $t_{b0}$ & 1h\\
Liczba dużych maszyn & $c_0$ & [0 - 10]\\
Wypłata operatora małej maszyny & $s_1$ & 60\\
Ilość surowca na mały element & $m_1$ & 3\\
Czas przygotowania małej maszyny & $t_{p1}$ & 45 min\\
Wartość małego elementu & $v_{b1}$ & 35\\
Czas wytworzenia małego elementu & $t_{b1}$ & 45 min\\
Ilość małych maszyn & $c_1$ & [0 - 10]\\
Maksymalny czas pracy pracownika & $t_w$ & 8h\\
Bonus pracowniczy & b & [0.0 - 0.2]\\
Współczynnik kary & $p_{un}$ & 1.5
\end{longtable}

Gdzie:
\begin{itemize}
    \item $x$ -- ilość wymaganych elementów * koszt części 
    \item Parametry wejściowe podane są w kwadratowych nawiasach
    \item Pracownik jest zatrudniony na pełen etat (8h płacone z góry)
    \item Pierwsza i druga zmiana są identyczne w ilość i rodzaj maszyn i pracowników
    \item Rezerwujemy surowce na wymagane elementy
    \item Wszystkie elementy ponad wymaganą liczbę są czystym dochodem
\end{itemize}


\section{Badany problem}
\subsection{Przegląd literatury}
This article is about the Clonal Selection Algorithm used for Optimalization in Electromagnetics. The authors present their own concept of real-coded clonal selection algorithm that can be used in electromagnetic design optimization. The article describes in detail all the algorithm parameters as well as the operation of the algorithm for "The TEAM Workshop problem 22”\cite{1430953}.



Artykuł przedstawia zastosowanie sztucznego systemu immunologicznego w aplikacji przemysłowej. Na postawie parametrów obróbki (siła, moment, itp.) oraz zakłócenia (wibracje, itp.) autorzy wykrywają uszkodzenie narzędzia. Wykorzystywany jest algorytm sztucznego systemu immunologicznego wykorzystuje do działania algorytm selekcji negatywnej \cite{dasgupta1999artificial}.


Artykuł przedstawia użycie algorytmów sztucznych systemów immunologicznych w przemyśle. Porównuje on algorytmy sztucznej inteligencji z algorytmem klonowania do algorytmu z mechanizmem uczenia społecznego. Zmieniając wzmocnienie, czas zdwojenia oraz czas wyprzedzenia dobierają one nastawy regulatora PID \cite{wang_artificial_2017}.


This article is about the Clonal Selection Based Memetic Algorithm for Job Shop Scheduling Problems. The authors' goal was to improve exploration and exploitation using a clonal selection algorithm. The article presents the use of clonal selection to construct an evolutionary search mechanism that is used for exploration\cite{yang2008clonal}.

Artykuł przedstawia użycie Algorytmu Selekcji Klonalnej w zastosowaniach inżynierskich. Opisane w nim jest działanie algorytmu od strony teoretycznej, a także działanie zaimplementowanego przez autorów algorytmu przy rozwiązywaniu trzech różnych problemów: binarnego rozpoznawania znaków, wielomodalnej optymalizacji funkcji -
$f(x, y) = x.\sin{4 \pi x} - y.\sin{4 \pi y + \pi} + 1$
i problemu komiwojażera dla 30 miast \cite{de_castro}.


Artykuł przedstawia użycie Algorytmu Selekcji Klonalnej do optymalizacji ułożenia terenu budowy. Zaprezentowany algorytm minimalizuje koszty produkcji i dystans przebyty pomiędzy n obiektami zaprezentowanymi za pomocą macierzy permutacji o wymiarach n x n \cite{WANG2016267}.


Artykuł przedstawia działanie sztucznego systemu immunologicznego (AIS) w przypadku rozwiązania pojemnościowego problemu marszrutyzacji. Celem było znalezienie odpowiedniego zestawienia parametrów algorytmu selekcji klonalnej w celu rozwiązania problemu poprzez podejście eksperymentalne. W artykule oprócz działania AIS, opisano także działanie innych metod rozwiązujących dwadzieścia instancji problemu i przedstawiono wyniki pod względem jakości rozwiązań oraz wykorzystanego czasu obliczeniowego \cite{thapatsuwan}.


Artykuł dotyczy zastosowania algorytmu selekcji klonalnej w celu określenia optymalnych punktów pracy w niskonapięciowych, hybrydowych mikrosieciach AC/DC. Celem było zminimalizowanie strat mocy czynnej, kosztów eksploatacji oraz optymalizacja napięcia węzłowego \cite{rokicki}.


\section{Diagram UML fabryki}
\begin{figure}[ht]
\centering
\includegraphics[width=.7\textwidth]{UML_Model.png}
\caption{Diagram UML}
\end{figure}


\newpage
\printbibliography

\end{document}

